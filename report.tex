\documentclass{article}

\usepackage{amsthm}
\usepackage{amsfonts}
\usepackage{amsmath}
\usepackage{amssymb}
\usepackage{fullpage}
\usepackage[usenames]{color}
\usepackage{hyperref}
  \hypersetup{
    colorlinks = true,
    urlcolor = blue,       % color of external links using \href
    linkcolor= blue,       % color of internal links
    citecolor= blue,       % color of links to bibliography
    filecolor= blue,        % color of file links
    }

\usepackage{listings}
\usepackage{graphicx}
\graphicspath{ {./Report Images/} }

\definecolor{dkgreen}{rgb}{0,0.6,0}
\definecolor{gray}{rgb}{0.5,0.5,0.5}
\definecolor{mauve}{rgb}{0.58,0,0.82}

\lstset{frame=tb,
  language=haskell,
  aboveskip=3mm,
  belowskip=3mm,
  showstringspaces=false,
  columns=flexible,
  basicstyle={\small\ttfamily},
  numbers=none,
  numberstyle=\tiny\color{gray},
  keywordstyle=\color{blue},
  commentstyle=\color{dkgreen},
  stringstyle=\color{mauve},
  breaklines=true,
  breakatwhitespace=true,
  tabsize=3
}

\theoremstyle{theorem}
   \newtheorem{theorem}{Theorem}[section]
   \newtheorem{corollary}[theorem]{Corollary}
   \newtheorem{lemma}[theorem]{Lemma}
   \newtheorem{proposition}[theorem]{Proposition}
\theoremstyle{definition}
   \newtheorem{definition}[theorem]{Definition}
   \newtheorem{example}[theorem]{Example}
\theoremstyle{remark}
  \newtheorem{remark}[theorem]{Remark}


\title{CPSC-354 Report}
\author{Marc Domingo  \\ Chapman University}

\date{\today}

\begin{document}

\maketitle
\tableofcontents

\section{Introduction}\label{introduction}
\subsection{Week 1}
\hspace{10mm} During the first week of Programming Languages, the class lectures went over the concept of \textit{Imperative vs Functional Programming} and \textit{Recursive Programming} in addition to a brief introduction to the programming language \textbf{Haskell}.
\subsection{Week 2}
\hspace{10mm} During the second week of Programming Languages, the class lectures went more into depth regarding techniques in how to use \textbf{Recursion} to solve problems, as well as introduced the concept of creating an \textit{Interpreter} in Haskell.
\subsection{Week 3} During the third week of Programming Languages, the class lectures went over the concept of Context and Parse Trees in relation to programming a calculator interpreter.
\subsection{Week 4} During the fourth week of Programming Languages, the lectures covered the concept of Lambda Calculus, specifically about its syntax and how to parse Lambda equations.
\subsection{Week 5} During the fifth week of Programming Languages, the lectures covered the concept of performing substitution in Lambda Calculus through pen-and-paper, and through an interpreter.
\subsection{Week 6} During the sixth week of Programming Languages, the languages covered the concept of how to substitute and rewrite variables within functions, before applying the concept to Lambda Calculus.
\subsection{Week 7} During the seventh week of Programming Languages, the lectures provided an introduction to the concept of Abstract Reduction Systems (ARS).
\subsection{Week 8} During the eighth, week of Programming Languages, the lectures covered the concepts of Confluence and Termination in how they relate to ARS.
\subsection{Week 9} During the ninth week of Programming Languages, the lectures covered the concepts of how to extend the definitions of Lambda calculus to if-then-else statements, defining variables through \textit{let}, and recursion, in addition to starting Assignment 2.
\subsection{Week 10} During the tenth week of Programming Languages, the lectures covered the concepts of Operational and Denotational Semantics, and how they apply to Natural Numbers. In addition to this, the lecture also involved the discussion about the design of a domain-specific language within the context of financial engineering.
\subsection{Week 11} During the eleventh week of Programming Languages, the lectures covered the introduction to Part 2 of Assignment 2, where we implement the concept of Lists into our Lambda Calculus language. The lectures also introduced the concept of Hoare Logic.
\subsection{Week 12} During the twelfth week of Programming Languages, the lectures covered the concept of Finitely Branching Systems, and eventually its application to ARS and proofs.
\section{Homework}\label{homework}
\subsection{Week 1}
The objective for this week's homework is to implement code to calculate the Greatest Common Denominator between two integers in a coding language of our choice.
\\
C++ Code for Greatest Common Denominator:
\begin{lstlisting}
#include <iostream>
using namespace std;

int gcd(int a, int b)
{
    if (a == 0)
    {
        return b;
    }

    if (b == 0)
    {
        return a;
    }

    if (a == b)
    {
        return a;
    }

    if (a > b)
    {
        return gcd((a - b), b);
    }

    if (b > a)
    {
        return gcd(a, (b - a));
    }
}

int main() {
    // Write C++ code here
    cout << "GCD of 9 and 33 is : " << gcd(9, 33);
    return 0;
}
\end{lstlisting}

    The function for calculating the Greatest Common Divisor (GCD) functions by first taking two integers as inputs, represented by \textbf{a} and \textbf{b}. In the case of inputs like gcd(9, 33), 9 and 33 are considered to be \textbf{a} and \textbf{b} in the gcd function respectively. The function first checks to see if either integer is \textit{0}, and in the case of either integer being zero, returns the other integer entered in the function as the gcd. If neither number is zero, the function checks to see if both integers are \textbf{equal to each other}. In the case that both integers are equal, the function returns the first integer as the gcd. If neither of the previous cases are met, the function then compares integer \textbf{a} and \textbf{b}. If integer \textbf{a} is larger, the function recursively calls itself, and \textit{integer a is replaced with (a - b)}. If integer \textbf{b} is larger, the function recursively calls itself, and \textit{integer b is replaced with (b - a).} The function \textit{continues making recursive calls} with modified numbers until a case where gcd is found.

\subsection{Week 2}
The objective of this week's homework is to implement the following functions in Haskell code--a function \textit{len} that takes in a list and calculates the length of a list, a function \textit{select\textunderscore evens} that takes in a list and returns only the even-indexed elements of that list, a function \textit{select\textunderscore odds} that takes in a list and returns only the odd-indexed elements of that list, a function \textit{member} that takes in an element of a list and a list and returns whether that element exists within the list, a function \textit{append} that takes in two lists and adds the elements of the second list to the first list, a function \textit{revert} that takes in a list and returns a list with its elements in the reverse order of the original list, and a function \textit{less\textunderscore equal} that takes in two lists and returns whether the first list is less than the second list or not.
\begin{lstlisting}
len [] = 0
len (x:xs) = 1 + len xs

select_evens [] = []
select_evens [a] = []
select_evens (x:y:list) = y:(select_evens list)

select_odds [] = []
select_odds [a] = [a]
select_odds (x:y:list) = x:(select_odds list)

member _ [] = False
member n (x:xs)
  | x == n = True
  | otherwise = member n xs

append [] list_original = list_original
append (x:list_add) list_original = x:(append list_add list_original)

revert [] = []
revert (item:xs) = append (revert xs) [item]

less_equal [] [] = True
less_equal (x:list_one) (y:list_two) = if x <= y
									    then less_equal list_one list_two
									    else False
\end{lstlisting}
For this section of the homework, the objective is to show the steps that Haskell would take in order to achieve the result of a function given specific inputs.
\\
\\
In the case of \textbf{select\textunderscore evens ["a","b","c","d","e"]}:
\begin{lstlisting}
select_evens ["a","b","c","d","e"] =
      "b":(select_evens ["c","d","e"]) =
      "b":"d":(select_evens ["e"]) =
      "b":"d":[] =
      ["b","d"]
\end{lstlisting}
In the case of \textbf{select\textunderscore odds ["a","b","c","d","e"]}:
\begin{lstlisting}
select_odds ["a","b","c","d","e"] =
      "a":(select_odds ["c","d","e"]) =
      "a":"c":(select_odds ["e"]) =
      "a":"c":["e"] =
      ["a","c","e"]
\end{lstlisting}
In the case of \textbf{member 2 [5,2,6]}:
\begin{lstlisting}
member 2 [5,2,6] =
        member 2 [2,6] =          5 != 2
        True                      2 == 2
\end{lstlisting}
In the case of \textbf{member 3 [5,2,6]}:
\begin{lstlisting}
member 3 [5,2,6] =
      member 3 [2,6] =          5 != 3
      member 3 [6] =            2 != 3
      member 3 [] =             6 != 3
      False
\end{lstlisting}
In the case of \textbf{append [1,2] [3,4,5]}:
\begin{lstlisting}
append [1,2] [3,4,5] =
      1:(append [2] [3,4,5]) =
      1:(2:(append [] [3,4,5])) =
      1:(2:([3:4:5])) =
      [1,2,3,4,5]
\end{lstlisting}
In the case of \textbf{revert [1,2,3]}:
\begin{lstlisting}
revert [1,2,3] =
      append (revert [2,3]) [1] =
      append (append (revert [3]) [2]) [1] =
      append (append (append (revert []) [3]) [2]) [1] =
      append (append (append [] [3]) [2]) [1] =
      append (append ([3]) [2]) [1] =
      append (3:(append [] ([2]))) [1] =
      append (3:([2]) [1] =
      append [3,2] [1] =
      3:(append [2] [1]) =
      3:(2:(append [] [1])) =
      3:(2:([1])) =
      [3,2,1]
\end{lstlisting}
In the case of \textbf{less\textunderscore equal [1,2,3] [2,3,4]}:
\begin{lstlisting}
less_equal [1,2,3] [2,3,4] =
      less_equal [2,3] [3,4] =          1 <= 2
      less_equal [3] [4] =              2 <= 3
      less_equal [] [] =                3 <= 4
      True
\end{lstlisting}
In the case of \textbf{less\textunderscore equal [1,2,3] [2,3,2]}:
\begin{lstlisting}
less_equal [1,2,3] [2,3,2] =
      less_equal [2,3] [3,2] =          1 <= 2
      less_equal [3] [2] =              2 <= 3
      False                             3 > 2
\end{lstlisting}

\subsection{Week 3}
The objective of this week's homework was to complete the function Hanoi, which represents the Tower of Hanoi puzzle involving a starting tower of 5 disks with the objective of moving the tower to the rightmost column, denoted by "2" in the last argument of the function.
\begin{lstlisting}
hanoi 5 0 2
    hanoi 4 0 1
        hanoi 3 0 2
            hanoi 2 0 1
                hanoi 1 0 2 = move 0 2
                move 0 1
                hanoi 1 2 1 = move 2 1
            move 0 2
            hanoi 2 1 2
                hanoi 1 1 0 = move 1 0
                move 1 2
                hanoi 1 0 2 = move 0 2
        move 0 1
        hanoi 3 2 1
            hanoi 2 2 0
                hanoi 1 2 1 = move 2 1
                move 2 0
                hanoi 1 1 0 = move 1 0
            move 2 1
            hanoi 2 0 1
                hanoi 1 0 2 = move 0 2
                move 0 1
                hanoi 1 2 1 = move 2 1
    move 0 2
    hanoi 4 1 2
        hanoi 3 1 0
            hanoi 2 1 2
                hanoi 1 1 0 = move 1 0
                move 1 2
                hanoi 1 0 2 = move 0 2
            move 1 0
            hanoi 2 2 0
                hanoi 1 2 1 = move 2 1
                move 2 0
                hanoi 1 1 0 = move 1 0
        move 1 2
        hanoi 3 0 2
            hanoi 2 0 1
                hanoi 1 0 2 = move 0 2
                move 0 1
                hanoi 1 2 1 = move 2 1
            move 0 2
            hanoi 2 1 2
                hanoi 1 1 0 = move 1 0
                move 1 2
                hanoi 1 0 2 = move 0 2
\end{lstlisting}
Movement solution for Towers of Hanoi with n Disks = 5:
\begin{lstlisting}
0->2
0->1
2->1
0->2
1->0
1->2
0->2
0->1
2->1
2->0
1->0
2->1
0->2
0->1
2->1
0->2
1->0
1->2
0->2
1->0
2->1
2->0
1->0
1->2
0->2
0->1
2->1
0->2
1->0
1->2
0->2
\end{lstlisting}
Within the computation for the Towers of Hanoi with \textit{n Disks = 5}, the word "Hanoi" appears a total of \textbf{31 times}. This can be expressed with the following equation:
$ Hanoi = 2^N - 1$, where \textit{N} is the number of disks from the initial tower.

\subsection{Week 4}
The objective of this week's homework is to explore the parser and abstract syntax tree of arithmetic. \\
The Abstract Syntax Tree of $2 + 1$ is:

\includegraphics[width=15cm, height=8cm]{Report Images/week4_1.jpg}
\begin{lstlisting}
Plus (Num 2) (Num 1)
\end{lstlisting}


The Abstract Syntax Tree of $1 + 2 * 3$ is:

\includegraphics[width=15cm, height=8cm]{Report Images/week4_2.jpg}
\begin{lstlisting}
Plus (Num 1) (Times (Num 2) (Num 3))
\end{lstlisting}

The Abstract Syntax Tree of $1 + (2 * 3)$ is:

\includegraphics[width=15cm, height=8cm]{Report Images/week4_3.jpg}
\begin{lstlisting}
Plus (Num 1) (Times (Num 2) (Num 3))
\end{lstlisting}

The Abstract Syntax Tree of $(1 + 2) * 3$ is:

\includegraphics[width=15cm, height=8cm]{Report Images/week4_4.jpg}
\begin{lstlisting}
Times (Plus (Num 1) (Num 2)) (Num 3)
\end{lstlisting}

The Abstract Syntax Tree of $1 + 2 * 3 + 4 * 5 + 6$ is:

\includegraphics[width=15cm, height=8cm]{Report Images/week4_5.jpg}
\begin{lstlisting}
Plus (Plus (Plus (Num 1) (Times (Num 2) (Num 3))) (Times (Num 4) (Num 5))) (Num 6)
\end{lstlisting}


The abstract syntax tree of $1+2+3$ would be identical to the one of $(1+2)+3$ as the compiler would calculate from left to right as there are no other higher-order operations aside from addition.

\subsection{Week 5}
The objective of this week's homework is to explore the parser for Lambda Calculus, as well as to explore the concept of Linearized Trees.\\
\textbf{Part 1:}

The Linearized Tree for $x$ is:
\begin{lstlisting}
x
Prog (EVar (Id "x"))
\end{lstlisting}

Its Abstract Syntax Tree:

\includegraphics[width=15cm, height=8cm]{Report Images/week5_1.jpg}

The Linearized Tree for x x is:
\begin{lstlisting}
x x
Prog (EApp (EVar (Id "x")) (EVar (Id "x")))
\end{lstlisting}

Its Abstract Syntax Tree:

\includegraphics[width=15cm, height=8cm]{Report Images/week5_2.jpg}

The Linearized Tree for x y is:
\begin{lstlisting}
x y
Prog (EApp (EVar (Id "x")) (EVar (Id "y")))
\end{lstlisting}

Its Abstract Syntax Tree:

\includegraphics[width=15cm, height=8cm]{Report Images/week5_3.jpg}

The Linearized Tree for x y z is:
\begin{lstlisting}
x y z
Prog (EApp (EApp (EVar (Id "x")) (EVar (Id "y"))) (EVar (Id "z")))
\end{lstlisting}

Its Abstract Syntax Tree:

\includegraphics[width=15cm, height=8cm]{Report Images/week5_4.jpg}

The Linearized Tree for \textbackslash  x.x is:
\begin{lstlisting}
\ x . x
Prog (EAbs (Id "x") (EVar (Id "x")))
\end{lstlisting}

Its Abstract Syntax Tree:

\includegraphics[width=15cm, height=8cm]{Report Images/week5_5.jpg}

The Linearized Tree for \textbackslash x.x x is:
\begin{lstlisting}
\ x . x x
Prog (EAbs (Id "x") (EApp (EVar (Id "x")) (EVar (Id "x"))))
\end{lstlisting}

Its Abstract Syntax Tree:

\includegraphics[width=15cm, height=8cm]{Report Images/week5_6.jpg}

The Linearized Tree for (\textbackslash x . (\textbackslash y . x y)) (\textbackslash x.x) z is:
\begin{lstlisting}
\ x . \ y . x y (\ x . x)z
Prog (EApp (EApp (EAbs (Id "x") (EAbs (Id "y") (EApp (EVar (Id "x")) (EVar (Id "y"))))) (EAbs (Id "x") (EVar (Id "x")))) (EVar (Id "z")))
\end{lstlisting}

Its Abstract Syntax Tree:

\includegraphics[width=15cm, height=8cm]{Report Images/week5_7.jpg}

The Linearized Tree for (\textbackslash x . \textbackslash y . x y z) a b c is:
\begin{lstlisting}
\ x . \ y . x y z a b c
Prog (EApp (EApp (EApp (EAbs (Id "x") (EAbs (Id "y") (EApp (EApp (EVar (Id "x")) (EVar (Id "y"))) (EVar (Id "z"))))) (EVar (Id "a"))) (EVar (Id "b"))) (EVar (Id "c")))
\end{lstlisting}

Its Abstract Syntax Tree:

\includegraphics[width=15cm, height=8cm]{Report Images/week5_8.jpg}

\textbf{Part 2:}

Evaluating of Equations:
\begin{lstlisting}
  (\x.x) a = a
  \x.x a = \x.x a
  (\x.\y.x) a b = (\y. a) b = a
  (\x.\y.y) a b = (\y. y) b = b
  (\x.\y.x) a b c = (\y. a) b c = (a) c = a c
  (\x.\y.y) a b c = (\y. y) b c = (b) c = b c
  (\x.\y.x) a (b c) = (\y. a) (b c) = a
  (\x.\y.y) a (b c) = (\y. y) (b c) = b c
  (\x.\y.x) (a b) c = (\y. a b) c = a b
  (\x.\y.y) (a b) c = (\y. y) c = c
  (\x.\y.x) (a b c) = (\y. a b c)
  (\x.\y.y) (a b c) = (\y. y)
\end{lstlisting}

\includegraphics[width=15cm, height=8cm]{Report Images/week5_9.jpg}

\subsection{Week 6}
The objective of this week's homework is to practice reducing Lambda Calculus equations and to write an outline of the final project for this class see "Specifiction". \\
\begin{lstlisting}
(\exp . \two . \three . exp two three)
(\m.\n. m n)
(\f.\x. f (f x))
(\f.\x. f (f (f x)))

=
((\m.\n. m n) (\f2.\x2. f2 (f2 x2)) (\f3.\x3. f3 (f3 (f3 x3))))
=
((\n. (\f2.\x2. f2 (f2 x2)) n) (\f3.\x3. f3 (f3 (f3 x3))))
=
((\f2.\x2. f2 (f2 x2)) (\f3.\x3. f3 (f3 (f3 x3))))
=
(\x2. (\f3.\x3. f3 (f3 (f3 x3))) ((\f3.\x3. f3 (f3 (f3 x3))) x2))
=
(\x2. (\f3.\x3. f3 (f3 (f3 x3))) (\x3. x2 (x2 (x2 x3))))
=
(\x2. (\x3. (\x3. x2 (x2 (x2 x3))) ((\x3. x2 (x2 (x2 x3))) ((\x3. x2 (x2 (x2 x3))) x3))))
=
(\x2. (\x3. (\x3. x2 (x2 (x2 x3))) ((\x3. x2 (x2 (x2 x3))) ((x2 (x2 (x2 x3)))))))
=
(\x2. (\x3. (\x3. x2 (x2 (x2 x3))) ((x2 (x2 (x2 (x2 (x2 (x2 x3)))))))))
=
(\x2. (\x3. (x2 (x2 (x2 (x2 (x2 (x2 (x2 (x2 (x2 x3)))))))))))
\end{lstlisting}
\subsection{Week 7}
The objective of this week's homework is to explore the concept of scope and bound in the case of program functions, as well as to practice more Lambda Calculus substitution and to explore the properties of Abstract Reduction Systems (ARS).
\\
For lines 5-8 as shown through:
\begin{lstlisting}
evalCBN (EApp e1 e2) = case (evalCBN e1) of
    (EAbs i e3) -> evalCBN (subst i e2 e3)
    e3 -> EApp e3 e2
evalCBN x = x
\end{lstlisting}
In \textit{Line 5}, \textit{e1} and \textit{e2} are bound to the left of \textit{=}, and the scope is the end of \textit{Line 7}. For \textit{Line 6}, the variables \textit{i} and \textit{e3} are bound to the left of \textit{->}, and the scope is the end of the same line. Meanwhile, in \textit{Line 7}, the variable \textit{e3} is bound to the left of \textit{->}, and its scope is the end of the same line. Finally, for \textit{Line 8}, the variable \textit{x} is bound to the left of \textit{=}, and its scope is the end of the same line.
\\
\\
For lines 18-22 as shown through:
\begin{lstlisting}
subst id s (EAbs id1 e1) =
    -- to avoid variable capture, we first substitute id1 with a fresh name inside the body of the Lambda-abstraction, obtaining e2. Only then do we proceed to apply substitution of the original s for id in the body e2.
    let f = fresh (EAbs id1 e1)
        e2 = subst id1 (EVar f) e1 in
        EAbs f (subst id s e2)
\end{lstlisting}
In \textit{Line 18}, the variables \textit{id}, \textit{s}, \textit{id1}, and \textit{e1} are all bound to the left of \textit{=}, and the scope is the end of \textit{Line 22}. Meanwhile, in \textit{Line 20}, the variable \textit{f} is bound to the left of \textit{=}, and the scope is also the end of \textit{Line 22}. Lastly, on \textit{Line 21}, the variable \textit{e2} is bound by \textit{=}, and its scope is the end of \textit{Line 22} as well.
\\
\\
Item 3:
\begin{lstlisting}
--Idea:
(\x.\y.x) y z = (\fresh.y) z = y

(EApp (EApp (EAbs (Id "x") (EAbs (Id "y") (EVar (Id "x")))) (EVar (Id "y"))) (EVar (Id "z")))

-- In first EApp, (EApp (EAbs (Id "x") (EAbs (Id "y") (EVar (Id "x")))) (EVar (Id "y"))) is e1, (EVar (Id "z")) is e2
-- In second EApp, (EAbs (Id "x") (EAbs (Id "y") (EVar (Id "x")))) is e1, (EVar (Id "y")) is e2

= -- Line 5
evalCBN ( ( evalCBN (EApp (EAbs (Id "x") (EAbs (Id "y") (EVar (Id "x")))) (EVar (Id "y"))) ) (EVar (Id "z")) )

    -- inner evalCBN
    -- Line 6
    evalCBN (subst ((EAbs (Id "x")) (EVar (Id "y")) (EAbs (Id "y") (EVar (Id "x")))))

        -- Line 18
        -- Line 20
        f = fresh ((EAbs (Id "y") (EVar (Id "x"))))
        -- consider \y now \y0
        -- Line 21
        e2 = subst (Id "y") (EVar "y0") (EVar (Id "x"))
        -- Line 16
        EAbs (Id "y0") (subst (Id "y") (EVar (Id "y")) (EVar (Id "x")))
        -- Line 16

    evalCBN (EAbs (Id "y0") (EVar (Id "y")))

= -- Line 7
evalCBN (EApp (EAbs (Id "y0")) (EVar (Id "y"))) (EVar (Id "z"))
= -- Line 6
evalCBN (subst (Id "y0") (EVar (Id "z")) (EVar (Id "y")))
= -- Line 16
evalCBN (EVar (Id "y"))
= -- Line 8
EVar (Id "y")
\end{lstlisting}
1. \textit{A = \{ \}}
\\ \includegraphics[width=15cm, height=8cm]{Report Images/week7_1.jpg}
\\2. \textit{A = \{a\}} and \textit{R = \{\}}
\\ \includegraphics[width=15cm, height=8cm]{Report Images/week7_2.jpg}
\\3. \textit{A = \{a\}} and \textit{R = \{(a,a)\}}
\\ \includegraphics[width=15cm, height=8cm]{Report Images/week7_3.jpg}
\\4. \textit{A = \{a, b, c\}} and \textit{R = \{(a,b), (a,c)\}}
\\ \includegraphics[width=15cm, height=8cm]{Report Images/week7_4.jpg}
\\5. \textit{A = \{a,b\}} and \textit{R = \{(a,a),(a,b)\}}
\\ \includegraphics[width=15cm, height=8cm]{Report Images/week7_5.jpg}
\\6. \textit{A = \{a, b, c\}} and \textit{R = \{(a,b),(b,b),(a,c)\}}
\\ \includegraphics[width=15cm, height=8cm]{Report Images/week7_6.jpg}
\\7. \textit{A = \{a, b, c\}} and \textit{R = \{(a,b),(b,b),(a,c),(c,c)\}}
\\ \includegraphics[width=15cm, height=8cm]{Report Images/week7_7.jpg}
\\
\\Example for cases
\\ 1. Confluent: True, Terminating: True, Unique Normal Forms: True
\\ \includegraphics[width=15cm, height=8cm]{Report Images/week7_8.jpg}
\\ 2. Confluent: True, Terminating: True, Unique Normal Forms: False
\\  No example of this would exist, as in order for an ARS to be Terminating and Confluent it would need an Unique Normal Form as if one of the normal forms of the initial term reduces to itself, while the other normal form reduces to that same normal form, the ARS would no longer be Terminating as it can infinitely reduce to itself.
\\ 3. Confluent: True, Terminating: False, Unique Normal Forms: True
\\ \includegraphics[width=15cm, height=8cm]{Report Images/week7_9.jpg}
\\ 4. Confluent: True, Terminating: False, Unique Normal Forms: False
\\ a points to itself, add picture later
\\ 5. Confluent: False, Terminating: True, Unique Normal Forms: True
\\ There is no example of this as in order for there to be a possibility of having an Unique Normal Form, the ARS must be at the very least Confluent, making "Confluence: False" an impossibility to have alongside "Unique Normal Form: False."
\\ 6. Confluent: False, Terminating: True, Unique Normal Forms: False
\\ \includegraphics[width=15cm, height=8cm]{Report Images/week7_11.jpg}
\\ 7. Confluent: False, Terminating: False, Unique Normal Forms: True
\\ There is no example of this as in order for an ARS to have a Unique Normal Form, at its base, the ARS must be at least Confluent, making "Confluence: False" an impossibility to have alongside "Unique Normal Form: True."
\\ 8. Confluent: False, Terminating: False, Unique Normal Forms: False
\\ \includegraphics[width=15cm, height=8cm]{Report Images/week7_7.jpg}
\subsection{Week 8}
The objective of this week's homework is to explore and analyze the properties of a specific Abstract Reduction System given its rules.\\
For the rewrite rules:
\begin{lstlisting}
  aa -> a
  bb -> b
  ba -> ab
  ab -> ba
\end{lstlisting}
\begin{enumerate}
    \item The ARS does not terminate because the output of the last rewrite rule ($ab\xrightarrow{}ba$) is the input of the third rewrite rule ($ba\xrightarrow{}ab$), and the output of the third rule is the input of the last rule. Because of instances such as: $ba \xrightarrow{} ab \xrightarrow{} ba ...$, the ARS can be rewritten infinitely and thus the ARS does not terminate.
    \item The normal forms of this ARS are \textit{a}, \textit{b}, and [ ]. [ ] is a normal form as if no characters are in the initial function, the function can only return the empty word [ ]. Meanwhile, the normal forms \textit{a} and \textit{b} can be seen in instances such as: $aaa \xrightarrow{} aa \xrightarrow{} a$ and $bbb \xrightarrow{}  bb \xrightarrow{} b$.
    \item The main change that would be made to the function would be to eliminate the fourth rule. Thus the ARS would have 3 normal forms: \textit{a}, \textit{b}, and \textit{ab}. This would then make the new ARS look like:
    \begin{lstlisting}
    aa -> a
    bb -> b
    ba -> ab
    \end{lstlisting}

    This maintains the same equivalence cases as the previous ARS in instances such as: $aaa \xrightarrow{} aa \xrightarrow{} a$ , $bbb \xrightarrow{}  bb \xrightarrow{} b$ , and [ ] $\xrightarrow{}$ [ ]. This does not change the equivalence of the original equation as by the definition of equivalence, the fourth rule $ab \xrightarrow{} ba$ can be also performed backwards $ab \xleftarrow{} ba$, which is the same rule as the third rule. In return, this results in another possible normal form \textit{ab}.
    \item The invariant of this algorithm is that the starting word: is empty \textit{[ ]}, contains at least one \textit{a}, contains at least one \textit{b}, or contains at least one \textit{a} and at least one \textit{b}. Through this, the specification of the algorithm determines that for any word, it determines if the word is \textit{blank}, contains the letter \textit{a} but not \textit{b}, contains the letter \textit{b} but not \textit{a}, or contains both the letters \textit{a} and \textit{b}.
\end{enumerate}
\subsection{Week 9}
The objective of this week's homework is to analyze an Abstract Reduction System (ARS) using the techniques and definitions that we have learned during lecture. \\
For the ARS (A, $\xrightarrow{}$) where A is the set of words over the alphabet ${a, b, c}$ and defined by the rules:
\begin{lstlisting}
ba -> ab
ab -> ba
ac -> ca
ca -> ac
bc -> cb
cb -> bc

aa -> b
ab -> c
ac ->
bb ->
cb -> a
cc -> b
\end{lstlisting}
The first six rules of the ARS indicate that the order of each member of the alphabet within a word does not matter, as each word can be rewritten into any other word containing the same letters. Consequently, the first six rules also show that the ARS does not necessarily terminate, due to being able to a potential to be rearranged infinitely. The last six rules of the ARS on the other hand, show that the ARS does actually have normal forms, which happen to be \textit{a}, \textit{b}, \textit{c}, and \textit{blank}. While the seventh, eighth, eleventh, and twelfth rules of the ARS indicate that if there are an even number of members within the starting word, that the ARS will result in a single member being the normal form of the ARS, the ninth and tenth rules of the ARS imply that in the event that the starting word has an odd number of members, that the resulting normal form of the ARS would be blank. Because of this, while the ARS may have normal forms, none of them are unique to the ARS. Because this ARS has multiple normal forms, it is not necessarily confluent either as the only case that it would be confluent is where the initial word has an odd number of members.
\subsection{Week 10}
The objective of this week's homework is to calculate Fix F 2, which represents the concept of a fixed point in calculus. \\
Given \textit{F} as:
\begin{lstlisting}
\f. \n. if n == 0 then 1 else f(n-1)*n
\end{lstlisting}
Reduction of $fix\textsubscript{F}2$, which is equivalent to $Ff\textsubscript{fix}2$:
\begin{lstlisting}
f2 =
    fixF 2 =
    F fixF 2 =
    -- Haskell is lazy and evaluates F first
    (\f. \n. if n == 0 then 1 else f(n-1) * n) FixF 2 =
    if 2 == 0 then 1 else fixF (2-1) * 2 =
    -- else case
    (fixF 1) * 2 =
    -- fixF 1 becomes F fixF 1, and Haskell evaluates F first
    (\f. \n. if n == 0 then 1 else f(n-1) * n) FixF 1 * 2=
    (if 1 == 0 then 1 else fixF (1-1) * 1) * 2=
    -- else case
    (((fixF 0) * 2) * 1) =
    -- 1 can be moved to the left-hand side, while fixF 0 becomes F fixF1, and Haskell evaluates F first
    (\f. \n. if n == 0 then 1 else f(n-1) * n) FixF 0 * 2 * 1=
    (if 0 == 0 then 1 else fixF(0) * 0) FixF -1 2 * 1=
    -- then case
    2 * 1 * 1 = -- arithmetic
    2
\end{lstlisting}
\subsection{Week 11}
For this section of the homework, the task was to write a 500-word essay regarding the topic of the application of DSL in the field of finances.
\\
\\
\indent Overall, the most interesting thing about learning about the application of a Domain Specific Programming Language (DSL) to finances, specifically the construction of increasingly complex contracts, was the fact that a lot of the syntax and grammar of the financial DSL was reminiscent of the syntax and grammar of functional programming languages, which shouldn’t have been surprising as the article had displayed a Haskell implementation of the financial DSL. \\

\indent Though a lot of confusion arose from the inclusion of mathematical symbols used to represent the contracts, the fact that we ourselves were confused by the symbols used to abstract various concepts relating to different aspects of contracts, made me question whether more financially savvy individuals would also decide to use the same symbolic representations for contracts or if it was more of an arbitrary choice made by those who had designed the DSL. Though I was informed that a knowledge gap for financial experts regarding having to learn the structure and functionality of this DSL would not impact the adoption of smart contracts in the field of finances as the existence of the DSL would already validate the potential of integrating smart contracts, I still feel as though there still exists potential for a knowledge gap that would have to be filled prior to mass integration of financial contracts. Instead of the knowledge gap existing for the financial experts, the gap would then exist for the software engineers who end up working with those financial experts. Given that our class, consisting of software engineering and computer science majors in addition to some of those students having some familiarity with finances, already had difficulty following some of the concepts introduced by the DSL, namely some of its abstractions of contract conditions, could potentially indicate that the software engineers who eventually begin working with financial experts using the DSL introduced in the article will likely need to have to have some prior curriculum relating to finance or training from another software engineer more savvy in using the DSL in order to effectively work with the financial expert in the event that the latter needs support when an unexpected error occurs while using a smart contract. \\

\indent I also found it interesting that the authors of the article had also mentioned that in addition to the Haskell implementation of the DSL shown in the article, that there also existed an implementation of the financial DSL through C++, and that the authors intend on using the success of the Haskell implementation to improve the C++ implementation. As C++ is not only a much more accessible programming language due to being more widely used and known in comparison to a functional programming language such as Haskell, in spite of not utilizing the more well-known object-oriented aspect of C++, it is capable of compiling at a greater rate than Haskell. Because of this, I was also curious about whether an Object-Oriented variation of Smart Contracts could potentially achieve similar success to the DSL variant as even though it would likely take longer to compile, is an abstraction method a lot more familiar among software engineers. In addition to this, an Object-Oriented variation of Smart Contracts could potentially be easier to extend in the event of Smart Contracts becoming increasingly more complex.
\\
\pagebreak
\subsection{Week 12}
The objective of this week's homework is to apply Hoare Logic to a given function.
\textbf{Given the equation: while (x!=0) do z:=z*y;  x:= x-1 done...} \\
\includegraphics[width=18cm, height=18cm]{Report Images/week12_2.jpg}
\\
\\ \includegraphics[width=18cm, height=20cm]{Report Images/week12.jpg}

\section{Project}\label{Project}
\subsection{Specification}
For my project in Programming Languages (Fall 2022), I intend on performing data analysis on a data set that is publicly available from GitHub. To accomplish this, the following would have to occur (in order):
\begin{enumerate}
    \item Receive approval from Dr. Kurtz for this specific project idea
    \item Come up with specific questions to answer about the Programming Languages data set, the responses to which can be modeled through various statistical models such as Linear/Logistic Regression Graphs, Data Cluster Graphs, Feature Reduction, LASSO Graphs, and/or Principle Component Analysis (PCA)
    \item Receive approval from Dr. Kurtz about the questions and the overall documentation of the data analysis report
    \item Download or import the data set from GitHub into Google Colab, this can be accomplished by replicating a specific iteration of the GitHub Programming Languages data set in an Excel sheet
    \item Using the Pandas package for Python, create statistical models to respond to each of the questions created for the report
    \item By the last day of October, provide a PDF of the Google Colab Notebook containing the compiled statistical models as a proof of implementation for this project. This can also be accomplished by embedding the Python code into the "Prototype" subsection of \textit{report.tex}, with each code block being above the respective model that it generated when compiled
    \item Write report for the analysis of the data set itself, explaining the process of selecting which model(s) to accept, reject, or if any model provided any surprising results about the data set in addition to providing a response to each of the questions created from \textbf{step 4}. A rough draft of this report should be submitted and present within the "Documentation" subsection of \textit{report.tex}
\end{enumerate}
\subsection{Prototype}
\textbf{Project Milestones}
\begin{enumerate}
    \item By \textbf{November 5, 2022}, email Dr. Kurtz the questions regarding the Programming Languages GitHub Data for the report, as well as the Excel sheet based on the data found from Githut.info.
    \begin{itemize}
        \item There should at least be \textbf{three} data analysis-related questions.
        \item The questions should relate to different types of statistical models, or involve the modification of a previously used statistical model using another data analysis technique.
        \item The questions should have some relation to the nature or ask for some conclusion about Programming Languages based on the resulting statistical model.
        \item This can alternatively be found uploaded to the \textit{Programming Languages Project Documents} folder on GitHub.
    \end{itemize}
    \item By \textbf{November 30, 2022}, create the statistical models associated for each of the questions.
    \begin{itemize}
        \item This will be accomplished via Google Colab
        \item The code and associated models for each of the statistical models should be included within the \textbf{Prototype} subsection of this report.
        \item Justifications for certain choices when generating the models (i.e. selecting a specific clustering methods or selecting which variables should be excluded from a revised model) should be also included alongside the code for the statistical models as comments.
        \item The Colab Notebook and a PDF of the compiled Notebook containing the statistical models should be uploaded to the \textit{Programming Languages Project Documents} folder on GitHub.
    \end{itemize}
    \item By \textbf{December 7, 2022}, have a rough draft of the responses to each of the questions for this report.
    \begin{itemize}
        \item This can be either written in a Word Document uploaded to the \textit{Programming Languages Project Documents} folder on GitHub or written directly to the \textbf{Documentation} section of this report.
        \item The responses to each of the questions, while still answering the questions, should contain a clear connection and come to a conclusion about trends and observations that can be made regarding the Programming Languages data.
        \item Justifications for decisions made regarding why certain modelling techniques or variables get excluded should also be reiterated within the report.
    \end{itemize}
\end{enumerate}
Link to the Jupyter Notebook containing the project itself: \href{https://colab.research.google.com/drive/1GuQKes41aCcwipScQs1ERPh6Ch4_szIW?usp=sharing}{GitHut Programming Languages Notebook}
\subsection{Documentation}
The dataset used for this project comes from \cite{GT} GitHut, a website designed with the intention of visualizing the data regarding the different Programming Languages associated with the various repositories of users on \cite{GH} GitHub. Although attempting to retrieve data from GitHub in general tends to be a difficult task, even according to some studies \cite{MS} not even appropriate for a task in spite of the data being relevant to a desired task at hand, there are still examples of studies and projects that have used GitHut data for analysis or visualization of that data. For instance, one particular study \cite{AA} used GitHub data regarding commits to and the contributors of Android App repositories in addition to download and review data from the Google Play store in order to create a visual connecting the two. Another study \cite{GD} had formed a dataset containing clones, copies of GitHub repositories that users can make their own personal modifications to, in addition to the original parent repository of the clones in order to study how cloned data can affect the performance of machine learning algorithms. Meanwhile, outside of repositories, other data from GitHub has been used in some projects, such as a recent article \cite{PP} where the objective was to create an algorithm that would recommend users potential partners to work with based on their recent activity on GitHub. Because of this, using the \cite{GT} GitHut data, I wanted to perform data analysis on the various Programming Languages statistics recorded on the site.
\\
\\
\indent Performing Linear Regression on each of the variables to predict the year in which a particular Programming Language on GitHub was released resulted in Linear Regression Models with similar R2 scores for the data used to train the models, but excluding the \textit{Opened Issues Per Repository} variable from the Linear Regression model resulted in an overall improvement in the testing R2 score and Mean Squared Error score. Conversely, removing the \textit{New Watchers Per Repository} from the Linear Regression resulted in an overall worse R2 and Mean Squared Error score for both the training and testing dataset. This had shown that while the former was overall not very indicative of a Programming Language's characteristics, the latter variable had a significant impact in identifying such Programming Languages. Performing Clustering on the dataset based on the \textit{Open Issues Per Repository} and \textit{New Watchers Per Repository} resulted in a Cluster in both the DBSCAN Clustering and Hierarchical Clustering Methods consisting of Programming Languages with that fell around the average for both variables. In the case of the DBSCAN method, the "noise" points were any Programming Language that were extreme outliers in either variable, while the Hierarchical Method had two separate clusters relating to the extreme ends for both of the variables. Finally, though both the LASSO Feature Reduction methods had resulted in worse performance than the Linear Regression model containing all of the variables from GitHut, the manual method of LASSO had reconfirmed the results of the initial Linear Regression model testing in that the \textit{New Watchers Per Repository} variable was the most significant variable in terms of identifying the characteristics of a particular Programming Language.
\subsection{Critical Appraisal}
\begin{itemize}
    \item Like the conclusions of some of the studies regarding the feasibility of GitHub data \cite{MS}, the GitHut dataset may not have been the best choice of data to form a narrative about, despite being most related to Programming Languages.
    \begin{itemize}
        \item It was overall very difficult to draw conclusions from purely numerical data without any categorical variables or significantly personal connections, such as when I had previously performed Data Analysis on an Unemployment and Mental Health dataset \cite{MH}.
    \end{itemize}
    \item Overall, one of the major limiting factors on the technical end of this project was attempting to perform all of the Data Science techniques and models for such a small dataset, as focusing on one particular quarter of the GitHut data only provided 30 data entries, which severely limited the amount of data that could be used to train and test the models
    \begin{itemize}
        \item A potential way to alleviate this in a future revisit of this project would be to record the GitHut data from the previous quarters in order to have more data to work with.
        \begin{itemize}
            \item From here, another potential way to tackle this dataset would to be observe how the rankings for each of the statistics on GitHub have changed over time (ie a particular Programming Language picking up popularity among users on GitHub would suddenly have the highest \textit{New Watchers Per Repository} or \textit{Active Repositories} value during a more recent quarter when it originally had a smaller value in both variables.
        \end{itemize}
    \end{itemize}
\end{itemize}
\section{Conclusions}\label{Conclusions}
\indent Without getting too deep in terms of detail, I would say that this course was overall important in that it gave insight into the various aspects of what goes into creating a programming language that we generally take for granted. In comparison to courses such as intro to Computer Science, where Chapman students learn how to program in Python, this course had the objective of having students understand the purpose and reasoning behind why some aspects of programming languages, such as specific things like spacing or grammar for example, can cause errors when attempting to compile that particular program. It was overall interesting to see the potential ways an individual who creates their own programming language can define the grammar of their language, as we have seen in both the Lambda Natural Numbers (Assignment 2) and Lambda Fun (Assignment 3) in class through the Lambda Calculus (.lc)  language that was used for those programming assignments. In addition to the logic behind front end aspects of programming languages such as syntax and grammar, the course was also significant in that it also taught us about aspects of computation such as the logic behind relatively basic concepts such as calculators (as we had covered this leading up to and during the Natural Number Calculator (Assignment 1) programming assignment) to more higher-level concepts such as the invariant of operations and functions. I would say that this particular aspect of the course relates the most to the wider world of software engineering as it relates more to the design and logic behind potential software that software engineers may need to decide on in the event that they attempt to improve upon what currently exists within the scope of computation or if they decide to create their own programming language in order to accomplish a specific goal such as protecting their intellectual secrets or to find a way to have their desired product run concisely without the unnecessary aspects of other programming languages.
\\
\indent In terms of learning Haskell throughout the course, it was interesting to learn about purely functional programming languages and the concept of pattern matching in place of simply having several if-then-else statements like we would typically use in other programming languages. Learning about functional programming languages relates to the wider world of software engineering as we have seen applications of using Haskell to create a Domain Specific Language specifically in the financial field in regard to creating contracts. In other words, it provided insight into understanding how to potentially create Domain Specific Languages for other fields and specific constructs.
\begin{thebibliography}{99}
\bibitem[AA]{AA} \href{https://dl.acm.org/doi/abs/10.1145/3196398.3196460}{A graph-based dataset of commit history of real-world Android apps}, Geiger et al. 2018.
\bibitem[GD]{GD} \href{https://dl.acm.org/doi/abs/10.1145/3379597.3387496}{A Dataset for GitHub Repository Deduplication}, Spinellis et al. 2020.
\bibitem[GH]{GH} \href{https://github.com}{GitHub}, GitHub, Inc. 2022.
\bibitem[GT]{GT} \href{githut.info}{GitHut}, Carlo Zapponi 2014.
\bibitem[MH]{MH} \href{https://www.kaggle.com/datasets/michaelacorley/unemployment-and-mental-illness-survey}{Unemployment and mental illness survey}, Corley, 2019.
\bibitem[MS]{MS} \href{https://gousios.org/pub/mining-soft-eng-data-github.pdf}{Mining Software Engineering Data from GitHub}, Gousios and Spinellis, 2017.
\bibitem[PP]{PP} \href{https://www.sciencedirect.com/science/article/abs/pii/S0950584922001033}{Find potential partners: A GitHub user recommendation method based on event data}, Bai et al. 2022.
\end{thebibliography}
\end{document}
