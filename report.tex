\documentclass{article}

\usepackage{amsthm}
\usepackage{amsfonts}
\usepackage{amsmath}
\usepackage{amssymb}
\usepackage{fullpage}
\usepackage[usenames]{color}
\usepackage{hyperref}
  \hypersetup{
    colorlinks = true,
    urlcolor = blue,       % color of external links using \href
    linkcolor= blue,       % color of internal links 
    citecolor= blue,       % color of links to bibliography
    filecolor= blue,        % color of file links
    }
    
\usepackage{listings}

\definecolor{dkgreen}{rgb}{0,0.6,0}
\definecolor{gray}{rgb}{0.5,0.5,0.5}
\definecolor{mauve}{rgb}{0.58,0,0.82}

\lstset{frame=tb,
  language=haskell,
  aboveskip=3mm,
  belowskip=3mm,
  showstringspaces=false,
  columns=flexible,
  basicstyle={\small\ttfamily},
  numbers=none,
  numberstyle=\tiny\color{gray},
  keywordstyle=\color{blue},
  commentstyle=\color{dkgreen},
  stringstyle=\color{mauve},
  breaklines=true,
  breakatwhitespace=true,
  tabsize=3
}

\theoremstyle{theorem} 
   \newtheorem{theorem}{Theorem}[section]
   \newtheorem{corollary}[theorem]{Corollary}
   \newtheorem{lemma}[theorem]{Lemma}
   \newtheorem{proposition}[theorem]{Proposition}
\theoremstyle{definition}
   \newtheorem{definition}[theorem]{Definition}
   \newtheorem{example}[theorem]{Example}
\theoremstyle{remark}    
  \newtheorem{remark}[theorem]{Remark}


\title{CPSC-354 Report}
\author{Marc Domingo  \\ Chapman University}

\date{\today}

\begin{document}

\maketitle
\section{Homework}\label{homework}
\subsection{Week 1}

C++ Code for Greatest Common Denominator:
\begin{lstlisting}
#include <iostream>
using namespace std;

int gcd(int a, int b)
{
    if (a == 0)
    {
        return b;
    }
    
    if (b == 0)
    {
        return a;
    }
    
    if (a == b)
    {
        return a;
    }
    
    if (a > b)
    {
        return gcd((a - b), b);
    }
    
    if (b > a)
    {
        return gcd(a, (b - a));
    }
}

int main() {
    // Write C++ code here
    cout << "GCD of 9 and 33 is : " << gcd(9, 33);
    return 0;
}
\end{lstlisting}
    
    The function for calculating the Greatest Common Divisor (GCD) functions by first taking two integers as inputs, represented by \textbf{a} and \textbf{b}. In the case of inputs like gcd(9, 33), 9 and 33 are considered to be \textbf{a} and \textbf{b} in the gcd function respectively. The function first checks to see if either integer is \textit{0}, and in the case of either integer being zero, returns the other integer entered in the function as the gcd. If neither number is zero, the function checks to see if both integers are \textbf{equal to each other}. In the case that both integers are equal, the function returns the first integer as the gcd. If neither of the previous cases are met, the function then compares integer \textbf{a} and \textbf{b}. If integer \textbf{a} is larger, the function recursively calls itself, and \textit{integer a is replaced with (a - b)}. If integer \textbf{b} is larger, the function recursively calls itself, and \textit{integer b is replaced with (b - a).} The function \textit{continues making recursive calls} with modified numbers until a case where gcd is found.

\subsection{Week 2}
\begin{lstlisting}
len [] = 0
len (x:xs) = 1 + len xs

select_evens [] = []
select_evens [a] = []
select_evens (x:y:list) = y:(select_evens list)

select_odds [] = []
select_odds [a] = [a]
select_odds (x:y:list) = x:(select_odds list)

member _ [] = False
member n (x:xs)
  | x == n = True
  | otherwise = member n xs

append [] _ = []
append list_original [] = list_original
append list_original (x:list_add) = append (list_original ++ [x]) list_add

revert [] = []
revert (item:xs) = (revert xs) ++ [item]

less_equal [] [] = True
less_equal (x:list_one) (y:list_two) = if x <= y
									    then less_equal list_one list_two
									    else False
\end{lstlisting}
In the case of \textbf{select\textunderscore evens ["a","b","c","d","e"]}:
$$
select\textunderscore evens \ ["a","b","c","d","e"] \xrightarrow{} "a":"b":["c","d","e"] = "b":(select\textunderscore evens \ ["c","d","e"])
$$
$$
"b":(select\textunderscore evens \ ["c","d","e"]) \xrightarrow{}
"b":("c":"d":["e"]) = "b":"d":(select\textunderscore evens \ ["e"])
$$
$$
"b":"d":(select\textunderscore evens \ ["e"]) \xrightarrow{} "b":"d":[]
$$
$$
"b":"d":[] \xrightarrow{} "b":["d"] \xrightarrow{} \textbf{["b","d"]}
$$
In the case of \textbf{select\textunderscore odds ["a","b","c","d","e"]}:
$$
select\textunderscore odds \ ["a","b","c","d","e"] \xrightarrow{} "a":"b":["c","d","e"] = "a":(select\textunderscore odds \ ["c","d","e"])
$$
$$
"a":(select\textunderscore odds \ ["c","d","e"]) \xrightarrow{}
"a":("c":"d":["e"]) = "a":"c":(select\textunderscore odds \ ["e"])
$$
$$
"a":"c":(select\textunderscore evens \ ["e"]) \xrightarrow{} "a":"c":["e"]
$$
$$
"a":"c":["e"] \xrightarrow{} "a":["c","e"] \xrightarrow{} \textbf{["a","c","e"]}
$$
In the case of \textbf{member 2 [5,2,6]}
$$
member \ 2 \ [5,2,6] \xrightarrow{} member \ 2 \ (5:[2,6]) \xrightarrow{} member \ 2 \ [2,6] \hspace{35pt} 5 != 2
$$
$$
member \ 2 \ [2,6] \xrightarrow{} member \ 2 \ (2:[6]) \xrightarrow{} \textbf{True} \hspace{35pt} 2 == 2
$$
In the case of \textbf{member 3 [5,2,6]}
$$
member \ 3 \ [5,2,6] \xrightarrow{} member \ 3 \ (5:[2,6]) \xrightarrow{} member \ 3 \ [2,6] \hspace{35pt} 5 != 3
$$
$$
member \ 3 \ [2,6] \xrightarrow{} member \ 3 \ (2:[6]) \xrightarrow{} member \ 3 \ [6] \hspace{35pt} 2 != 3
$$
$$
member \ 3 \ [6] \xrightarrow{} member \ 3 \ (6:[]) \xrightarrow{} member \ 3 \ [] \hspace{35pt} 6 != 3
$$
$$
member \ 3 \ [] \xrightarrow{} \textbf{False}
$$
In the case of \textbf{append [1,2] [3,4,5]}:
$$
append \ [1,2] \ [3,4,5] \xrightarrow{} append \ [1.2] \ (3:[4,5]) = append \ ([1,2] +\!+ \ 3) \ [4,5]
$$
$$
append ([1,2] +\!+ \ 3) \ [4,5] \xrightarrow{} append \ ([1,2] +\!+ \ 3) \ (4:[5]) = append \ ([1,2] +\!+ \ 3 +\!+ \ 4) \ [5]
$$
$$
append \ ([1,2] +\!+ \ 3 +\!+ \ 4) \ [5] \xrightarrow{}
append \ ([1,2] +\!+ \ 3 +\!+ \ 4) \ (5:[]) = append \ ([1,2] +\!+ \ 3 +\!+ \ 4 +\!+ \ 5) \ []
$$
$$
append \ ([1,2] +\!+ \ 3 +\!+ \ 4 +\!+ \ 5) \ [] = ([1,2] +\!+ \ 3 +\!+ \ 4 +\!+ \ 5)
$$
$$
([1,2] +\!+ \ 3 +\!+ \ 4 +\!+ \ 5) \xrightarrow{} ([1,2,3] +\!+ \ 4 +\!+ \ 5) \xrightarrow{} ([1,2,3,4] +\!+ \ 5) \xrightarrow{} \textbf{[1,2,3,4,5]}
$$
In the case of \textbf{revert [1,2,3]}:
$$
revert \ [1,2,3] \xrightarrow{} revert \ (1:[2,3]) = (revert \ [2,3]) +\!+ \ 1 
$$
$$
(revert \ [2,3]) +\!+ \ 1 \xrightarrow{} (revert \ 2:[3]) +\!+ \ 1 = (revert \ [3]) +\!+ 2 +\!+ \ 1
$$
$$
(revert \ [3]) +\!+ 1 +\!+ \ 2 \xrightarrow{} (revert \  3:[]) +\!+ \ 2 +\!+ \ 1 = (revert \ []) +\!+ \ 3 +\!+ \ 2 +\!+ \ 1
$$
$$
(revert \ []) +\!+ \ 3 +\!+ \ 2 +\!+ \ 1 = [] +\!+ \ 3 +\!+ \ 2 +\!+ \ 1
$$
$$
[] +\!+ \ 3 +\!+ \ 2 +\!+ \ 1 \xrightarrow{} [3] +\!+ \ 2 +\!+ \ 1 \xrightarrow{} [3,2] +\!+ \ 1 \xrightarrow{} \textbf{[3,2,1]}
$$
In the case of \textbf{less\textunderscore equal [1,2,3] [2,3,4]}:
$$
less\textunderscore equal \ [1,2,3] \ [2,3,4] \xrightarrow{} less\textunderscore equal \ (1:[2,3]) \ (2:[3,4]) = less\textunderscore equal \ [2,3] \ [3,4] \hspace{35pt} 1 \leq 2
$$
$$
less\textunderscore equal \ [2,3] \ [3,4] \xrightarrow{} less\textunderscore equal \ (2:[3]) \ (3:[4]) = less\textunderscore equal \ [3] \ [4] \hspace{35pt} 2 \leq 3
$$
$$
less\textunderscore equal \ [3] \ [4] \xrightarrow{} less\textunderscore equal (3:[]) \ (4:[]) = less\textunderscore equal \ [] \ [] \hspace{35pt} 3 \leq 4
$$
$$
less\textunderscore equal \ [] \ [] = \textbf{True}
$$
In the case of \textbf{less\textunderscore equal [1,2,3] [2,3,2]}:
$$
less\textunderscore equal \ [1,2,3] \ [2,3,2] \xrightarrow{} less\textunderscore equal \ (1:[2,3]) \ (2:[3,2]) = less\textunderscore equal \ [2,3] \ [3,2] \hspace{35pt} 1 \leq 2
$$
$$
less\textunderscore equal \ [2,3] \ [3,2] \xrightarrow{} less\textunderscore equal \ (2:[3]) \ (3:[2]) = less\textunderscore equal \ [3] \ [2] \hspace{35pt} 2 \leq 3
$$
$$
less\textunderscore equal \ [3] \ [2] \xrightarrow{} less\textunderscore equal (3:[]) \ (2:[]) = \textbf{False} \hspace{35pt} 3 > 2
$$

\subsection{Week 3}
\ldots
\end{document}